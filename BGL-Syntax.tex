\documentclass[11pt]{article}


% Times font
%
\usepackage{newtxtext,newtxmath}
\usepackage[american]{babel}
\renewcommand\ttdefault{cmtt}


% layout
%
\usepackage[top=1in, bottom=1in, left=1in, right=1in]{geometry}
\widowpenalty=0
\clubpenalty=0
\displaywidowpenalty=0
\raggedbottom
\sloppypar
\usepackage{needspace}



%%% Syntax macros
%%%
\newcommand{\prodName}[1]{&(\emph{#1})\\}
\newcommand{\nt}[1]{\textit{#1}}
\newcommand{\prog}[1]{\texttt{#1}}
\newcommand{\ORil}{\ \ |\ \ }
\newcommand{\IS}{&::=&}
\newcommand{\OR}{&|&}
\newcommand{\newsection}{\multicolumn{4}{l}{} \\}

\newcommand{\syntax}[2]{
  \needspace{3\baselineskip}
  \bigskip\par\noindent\textbf{#1}\par\smallskip%
  $\begin{array}{rcl@{\qquad}l}#2\end{array}$%
  \par\bigskip\noindent\ignorespaces
}

\newcommand{\lp}{\prog{(}}
\newcommand{\rp}{\prog{)}}
\newcommand{\paren}[1]{\lp#1\rp}
\newcommand{\pair}[2]{\paren{#1\prog{,}#2}}
\newcommand{\rbr}{\symbol{125}}
\newcommand{\bs}[1]{\symbol{92}}



\begin{document}

\title{BGL Syntax}
\author{Martin Erwig}
\maketitle

\noindent
%
We generally adopt the Haskell rules for lexical syntax, in particular,
\nt{name} is used for function and parameter names and stands for alphanumeric
strings starting with a lowercase letter.
%
Similarly, \nt{Name} is used for symbols (which are basically nullary
constructors) and stands for alphanumeric strings starting with an uppercase
letter. It is also used for the names of games.
%
The nonterminals \nt{int} and \nt{string} stand for integer and string
constants, respectively.
%
Comments are also handled as in Haskell.

A game definition consists of a name, two mandatory type definitions for the board and player inputs, and any number of value and function definitions.
%
The board definition determines the board dimensions and the type of values
contained on a board.

\syntax{Games and Type Definitions}{
\nt{game}   \IS  \prog{game}\ \nt{Name}\ \nt{board}\ \nt{input}\ \nt{valdef}
                                                     \prodName{game definition}
\nt{board}  \IS  \prog{type Board = Array }\pair{\nt{int}}{\nt{int}}\
                 \prog{of}\ \nt{type}                \prodName{board type}
\nt{input}  \IS  \prog{type Input =}\ \nt{type}      \prodName{input type}
}
%
Value definitions (for values and functions) require a signature declaring
their type plus an equation for defining the value. If a value is
parameterized, that is, if it is followed by a parameter list, it represents a
function.

A board definition (for a variable \nt{name} with declared type \prog{Board})
is equivalent to an array definition and may contain a number of definitions
for individual positions (given by pairs of integers) and for sets of
positions, which can be given by using a variable for either or both
coordinates. The semantics of a definition such as \prog{board(x,2) = Empty} is
to set all fields in the second row to \prog{Empty}.

\syntax{Value Definitions}{
\nt{valdef}    \IS \nt{signature}\ \nt{equation}   \prodName{definition}
\nt{signature} \IS \nt{name}\ \prog{:}\ \nt{type}  \prodName{type signature}
\nt{equation}  \IS \nt{name}\ \prog{=}\ \nt{expr}  \prodName{value equation}
               \OR \nt{name}\ \nt{parlist}\ \prog{=}\ \nt{expr}
                   \prodName{function equation}
               \OR \nt{boardequ}^*                  \prodName{board equation}
\nt{boardequ}  \IS \nt{name}\pair{\nt{pos}}{\nt{pos}}\ \prog{=}\ \nt{expr}
                   \prodName{board definition}
\nt{pos}       \IS \nt{int} \ORil \nt{name}      \prodName{board positions}
\nt{parlist} \IS \paren{\nt{name}_1\prog{,}\ \ldots\ \prog{,}\nt{name}_k}
                 \prodName{parameter list, $k\geq 1$}
}
%
The types system builds on a collection of basic types, which include
predefined numbers and booleans as well as the type \prog{Symbol}, which is the
type for all symbol values (\nt{Name}). The predefined symbol values \prog{A}
and \prog{B} also have the more specific type \prog{Player}. The special type
\prog{Input} is a synonym for the type of information that is gathered from the
user in each move during an execution of the game.
%
Basic types also include types specific to the domain of board games. Among
those, the type \prog{Position} is a synonym for \prog{(Int,Int)}. The type
\prog{Positions} is an abstract data type for collections of positions, which
can be though of as sets or lists of positions. However, the underlying
representation is not exposed to the programmer. A value of type
\prog{Positions} can only be generated and manipulated by a number of
predefined operations.

Extended types extend a base type by one or more symbols, which are values and
can be thought of nullary data constructors. For example, the Haskell type
\prog{Maybe Int} can be represented by the extended type \prog{Int|Nothing}.
Extended types facilitate the extension of types by values for representing
``special'' situations. For example, \prog{Player|Empty} is a type typically
used in a board type definition.

A tuple type can contain extended types, and a plain type is either a tuple
type or an extended type. The language has first-order function types whose
argument and result types can be any plain type.

\syntax{Types}{
\nt{btype}    \IS  \prog{Bool} \ORil \prog{Int} \ORil
                   \prog{Symbol} \ORil \prog{Input}
                                                         \prodName{atomic type}
              \OR  \prog{Board} \ORil \prog{Player} \ORil
                   \prog{Position} \ORil \prog{Positions}
                                                           \prodName{game type}
\nt{xtype}    \IS  \nt{btype}(\prog{|}\nt{Name})^*
                                                       \prodName{extended type}
\nt{ttype}    \IS  \paren{\nt{xtype}_1\prog{,}\ \ldots\ \prog{,}\nt{xtype}_k}
                                               \prodName{tuple type, $k\geq 2$}
\nt{ptype}     \IS  \nt{xtype} \ORil \nt{ttype}
                                                          \prodName{plain type}
\nt{ftype}    \IS  \nt{ptype}\ \prog{->}\ \nt{ptype}   \prodName{function type}
\nt{type}     \IS  \nt{ptype} \ORil \nt{ftype}
                                                                \prodName{type}
}
%
Atomic expressions are either basic integer or string values or symbols. Note
that symbols include the predefined values \prog{A} and \prog{B} of type
\prog{Player} plus a number of predefined operations.
%
Not that the case for infix application includes the Haskell notation for array
lookup \prog{board!(x,y)}.

\syntax{Expressions}{
\nt{expr}     \IS  \nt{int}                                 \prodName{integer}
              \OR  \nt{string}                               \prodName{string}
              \OR  \nt{Name}                                 \prodName{symbol}
              \OR  \nt{name}                               \prodName{variable}
            % \OR  \prog{A} \ORil \prog{B}                   \prodName{player}
              \OR  \paren{\nt{expr}}       \prodName{parenthesized expression}
              \OR  \paren{\nt{expr}_1\prog{,}\ \ldots\ \prog{,}\nt{expr}_k}
                                                    \prodName{tuple, $k\geq 2$}
%%% function application
              \OR  \nt{name}\paren{\nt{expr}_1\prog{,}\ \ldots\
                                              \prog{,}\nt{expr}_k}
                                               \prodName{function application}
              \OR  \nt{expr}\ \nt{binop}\ \nt{expr}
                                                  \prodName{infix application}
%%% bindings
             \OR  \prog{let}\ \nt{name}\ \prog{=}\ \nt{expr}\
                   \prog{in}\ \nt{expr}            \prodName{local definition}
%%% control strructures
              \OR  \prog{if}\ \nt{expr}\ \prog{then}\ \nt{expr}\
                   \prog{else}\ \nt{expr}               \prodName{conditional}
              \OR  \prog{while}\ \nt{expr}\ \prog{do}\ \nt{expr}
                                                         \prodName{while loop}
%
\nt{binop}    \IS  \prog{+} \ORil \prog{-} \ORil
                   \prog{==} \ORil \prog{!} \ORil \ldots
                                                   \prodName{binary operation}
}
%
% The following may not be needed for the syntax.
%
% \syntax{Predefined Operation Names}{
% \nt{predef} \IS  \prog{INPUT} \ORil \nt{redef}
%                  \prodName{predefined operations}
% \nt{redef}  \IS  \prog{gameOver} \ORil\ \prog{isValid}
%                  \prodName{redefinable operations}
% }
%
Here are several predefined operations and values and their types. This list is preliminary and is likely to change.

\begin{verbatim}
-- User input
input : Input

-- Board and Player updates
place : (Symbol,Board,Position) -> Board
next  : Player -> Player

-- Board predicates
free   : Board -> Positions
isFull : Board -> Bool
isFull(b) = countBoard(Empty,b) == 0

inARow      : (Int,Symbol,Board) -> Bool
countBoard  : (Symbol,Board) -> Int
countRow    : (Symbol,Board,Int) -> Int
countColumn : (Symbol,Board,Int) -> Int
\end{verbatim}


% \syntax{Auxiliary Definitions}{
% \nt{parlist} \IS \paren{\nt{name}_1\prog{,}\ \ldots\ \prog{,}\nt{name}_k}
%                  \prodName{parameter list, $k>1$}
% }


\end{document}
